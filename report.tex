%
% The first command in your LaTeX source must be the \documentclass command.
\documentclass[sigconf]{acmart}
\usepackage{graphicx} 
\usepackage{float} 
\usepackage{subfigure} 
\usepackage{listings}
\lstset{
 columns=fixed,       
 numbers=left,                                        
 numberstyle=\tiny\color{gray},                       
 frame=none,                                          
 backgroundcolor=\color[RGB]{245,245,244},             
 keywordstyle=\color[RGB]{40,40,255},                 
 numberstyle=\footnotesize\color{darkgray},           
 commentstyle=\it\color[RGB]{0,96,96},                
 stringstyle=\rmfamily\slshape\color[RGB]{128,0,0},   
 showstringspaces=false,                              
 language=c++,                                        
}
%
% defining the \BibTeX command - from Oren Patashnik's original BibTeX documentation.
\def\BibTeX{{\rm B\kern-.05em{\sc i\kern-.025em b}\kern-.08emT\kern-.1667em\lower.7ex\hbox{E}\kern-.125emX}}
    
%
% end of the preamble, start of the body of the document source.
\begin{document}

%
% The "title" command has an optional parameter, allowing the author to define a "short title" to be used in page headers.
\title{Comparative Research on Predictive Models Based on MOBA Game Data Set}


\settopmatter{printacmref=false} % Removes citation information below abstract
\renewcommand\footnotetextcopyrightpermission[1]{} % removes footnote with conference information in first column
\pagestyle{plain} % removes running headers


%
% The "author" command and its associated commands are used to define the authors and their affiliations.
% Of note is the shared affiliation of the first two authors, and the "authornote" and "authornotemark" commands
% used to denote shared contribution to the research.

\author{Yumin Xu, Michael Vigil, Logan Decker}
\affiliation{%
  \institution{New Mexico Institute of Mining and Technology\\
    Fall 2021 - CSE-589-Predictive Data Analytics}
  \streetaddress{801 Leroy Place}
  \city{Socorro}
  \state{New Mexico}}
\email{yumin.xu@student.nmt.edu}
\email{michael.vigil@student.nmt.edu}
\email{logan.decker@student.nmt.edu}


%
% The abstract is a short summary of the work to be presented in the article.
\begin{abstract}

\end{abstract}


%
% Keywords. The author(s) should pick words that accurately describe the work being
% presented. Separate the keywords with commas.
\keywords{ }


%
% This command processes the author and affiliation and title information and builds
% the first part of the formatted document.
\maketitle

\section{Introduction}

\section{Training Model}

\subsection{Decision Tree}

\subsubsection{Dota 2 Data Set }

\subsubsection{LoL Data Set}

\subsection{K-NN}

\subsubsection{Dota 2 Data Set }

\subsubsection{LoL Data Set}

\subsection{Naive Bayes}

\subsubsection{Dota 2 Data Set }

\subsubsection{LoL Data Set}

\section{Evaluating Model}

\subsection{Decision Tree}

\subsubsection{Dota 2 Data Set }

\subsubsection{LoL Data Set}

\subsection{K-NN}

\subsubsection{Dota 2 Data Set }

\subsubsection{LoL Data Set}

\subsection{Naive Bayes}

\subsubsection{Dota 2 Data Set }

\subsubsection{LoL Data Set}

\section{Comparing}

\subsection{Horizontal comparison}

\subsubsection{Decision Tree }

\subsubsection{K-NN}

\subsubsection{Naive Bayes}

\subsection{Longitudinal comparison}

\subsubsection{Dota 2 Data Set }

\subsubsection{LoL Data Set}

\section{Conclusion}

%
% The next two lines define the bibliography style to be used, and the bibliography file.
\bibliographystyle{ACM-Reference-Format}
\bibliography{sample-base}

\begin{thebibliography}{9}
\bibitem{c1} 
\bibitem{c1} 
\bibitem{c1} 
\bibitem{c1} 
\bibitem{c1} 
\bibitem{c1} 
\bibitem{c1}
\bibitem{c1} 
\bibitem{c1}
\bibitem{c1}
\bibitem{c1} 
\bibitem{c1} 


\end{thebibliography}

\end{document}
